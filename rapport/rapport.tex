\documentclass[french, 12pt]{article}

\usepackage[utf8]{inputenc}    % Encodage
\usepackage[T1]{fontenc}       % Encodage des polices
\usepackage{lipsum}            % Générateur de texte pour les exemples
\usepackage{graphicx}          % Pour inclure des images
\usepackage{amsmath}           % Pour les formules mathématiques
\usepackage{amsfonts}          % Pour les polices mathématiques
\usepackage{amssymb}           % Pour les symboles mathématiques
\usepackage{hyperref}          % Pour les liens
\usepackage{fancyhdr}          % Pour les en-têtes et pieds de page

% Configuration de l'en-tête et du pied de page
\pagestyle{fancy}
\fancyhf{}
\fancyhead[L]{Titre du Rapport}
\fancyhead[R]{Sacha David et Nathael Carlier}        
\fancyfoot[C]{\thepage}        

% Titre et auteur
\title{Project Sokoban}
\author{Sacha David et Nathael Carlier\\
        Prép'Isima 2}
\date{\today}




% détaillant la modélisation du jeu (choix retenus pour la modélisation du plateau, 
%des élémentsainsi que les strutures de données. À des fins d’évaluations, il sera précisé comment lancer leprogramme.)
\begin{document}

\maketitle

\tableofcontents

\section{Introduction}
% Explication du principe avec image

\section{Nécessaire pour le lancement du jeu}
% Les downloads nécessaire et comment le lancer

\section{Choix de  modélisation} % Ou D.A. jsp pas quoi mettre 
% Explication de pk un capy 

\section{Structuration des données}

    \subsection{Plateau de jeu}
    % Comment est gérer le plateau

    \subsection{Joueur}
    % Comment est gérer le joueur


\section{Affichage du jeu}
% Comment on affiche le tout 

\section{Conclusion}



\end{document}
